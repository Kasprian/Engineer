\chapter{Wstęp}
\thispagestyle{chapterBeginStyle}

We współczesnym nam świecie istnieje ciągła potrzeba tworzenia ofert i polecania produktów różnym klientom. Proste mechanizmy takie jak lista bestsellerów w księgarni internetowej, czy lista nowych produktów w sklepie z odzieżą stają się niewystarczające. Celem zwiększenia swoich szans w walce z konkurencją oraz zmaksymalizowania dochodów serwisy internetowe zaczęły rozglądać się za sposobem na ustalenie gustu indywidualnego klienta.

Z pomocą przyszły systemy rekomendacji, które pozwalają na stworzenie oferty spersonalizowanej pod wybranego użytkownika. Sprawdziły się one w swoim zadaniu tak efektywnie, że korzysta z nich większość serwisów internetowych. W branży e-handlu (ang. \textit{e-commerce}) Amazon stał się liderem sprzedaży stosując takie systemy przy polecaniu aukcji, serwisy hostujące wideo oraz muzykę np. Youtube czy Spotify nie byłyby tak efektywne w proponowaniu kolejnych filmów czy utworów, a wyszukiwanie znajomych na portalu społecznościowym takim jak Facebook, który w sierpniu 2019 roku przekroczył 2.45 miliarda aktywnych użytkowników \cite{facebook} byłoby po prostu niemożliwe.

Główną ideą stojącą za silnikami rekomendacji jest znalezienie pewnego podobieństwa między użytkownikiem, a produktami lub innymi użytkownikami, a następnie stworzenie rankingu, który je porówna. Ponieważ są one tak szeroko wykorzystywane, istnieje wielu różnych podejść oraz implementacji. Dwa główne, które możemy wyróżnić to metoda Content-based filtering, która polega na polecaniu produktów na podstawie tego, czym użytkownik interesował się w przeszłości oraz metoda Collaborative filtering, w której przy rekomendacji wykorzystujemy informacje jakie produkty podobały się użytkownikom o podobnym guście \cite{RecommenderASurvey}. Wszelkie inne stanowią najczęściej kombinacje obu technik z zamiarem dopasowania systemu do konkretnego problemu.
%JL lub rozwiązania problemu dotykającą, którąś z powyższych koncepcji.

Celem pracy jest porównanie technik wykorzystywanych do tworzenia systemu rekomendacji, a następnie stworzenie takiego systemu 
%JL w oparciu o błądzenie losowe 
w kontekście problemu przydzielania recenzentów do oceniania prac naukowych.

Praca składa się z czterech rozdziałów.
W rozdziale pierwszym znajduje się opis oraz klasyfikacja systemów rekomendacji, 
a w szczególności wprowadzenie do techniki opartej na spacerach losowych.


W rozdziale drugim przedstawiono opis stworzonego systemu.

W rozdziale trzecim znajdują się testy zaimplementowanego systemu.

W rozdziale czwartym przedstawiono sposób instalacji i wdrożenia systemu w środowisku docelowym.

Końcowy rozdział stanowi podsumowanie uzyskanych wyników.